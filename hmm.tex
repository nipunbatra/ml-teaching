\documentclass{beamer}
\usepackage{tcolorbox}
\newcommand{\red}[1]{{\color{red}\small \textbf{Apoorv Task:} #1}}



\usepackage{tikz}
\usetikzlibrary{arrows,backgrounds,quotes,arrows.meta, shapes.geometric}
\usepgflibrary{shapes.multipart}

\usepackage{mathtools}
\DeclarePairedDelimiter\ceil{\lceil}{\rceil}
\DeclarePairedDelimiter\floor{\lfloor}{\rfloor}

%\beamerdefaultoverlayspecification{<+->}
\newcommand{\data}{\mathcal{D}}

\DeclareMathOperator*{\argmin}{arg\,min}

\newcommand\Item[1][]{%
	\ifx\relax#1\relax  \item \else \item[#1] \fi
	\abovedisplayskip=0pt\abovedisplayshortskip=0pt~\vspace*{-\baselineskip}}


\usetheme{metropolis}           % Use metropolis theme
\tikzstyle{nodecircle} = [circle, draw=black, fill=green, text centered, radius=2cm]

\title{Hidden Markov Model (HMM)}
\date{\today}
\author{Nipun Batra}
\institute{IIT Gandhinagar}




\begin{document}
	\maketitle
	
	\begin{frame}{Markov Chains}
	\begin{itemize}
		\item Let us use a random variable $X_t$ to denote the observation of weather at day $t$
		\item Let us assume that there are only three types of weather - hot (H), cold (C), and warm (W)
		\item Let us draw a sequence of observations: $X_{1:T}$
	
		\item Markov Assumption: that weather tomorrow depends only on weather today
		\item $p(X_t|X_1, X_2, \cdots, X_{t-1}) = p(X_t|X_{t-1})$
		\item Given the present, the past and the future are independent
	\end{itemize}
	
	
	\end{frame}
	
	\begin{frame}{Markov Chains}
	\begin{tikzpicture}[node distance=2cm]
	\node (X1) (nodecircle) {$X_1$};
	\end{tikzpicture}

	\end{frame}

	%\item How will be the weather tomorrow if it was cold today, i.e. $P(X_{t+1}=Warm|X_{t}=Warm)$?

\end{document}